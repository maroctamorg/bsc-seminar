\begin{frame}
\frametitle{Groups, Rings and Fields}
\only<1>{
    \begin{definition}[Group]
        A group is a set closed under an associative, invertible product.
    \end{definition}
}
\only<1>{
    \begin{block}{Examples}
        $(\mathbb{Z}, +), (\mathcal{D}_{2n}, \circ), (\mathrm{GL}(n, \R), \cdot)$
    \end{block}
}
\only<2>{
    \begin{definition}[Ring]
        A ring is an Abelian group (we refer to the group operation as addition) equipped with a second associative and distributive binary operation (we refer to it as multiplication).
    \end{definition}
}
\only<2>{
    \begin{block}{Examples}
        $(\mathbb{Z}, +, \cdot), (\mathbb{Z}_n, +\pmod{n}, \cdot \pmod{n}), (\R[X], +, \cdot)$
    \end{block}
}

\only<3>{
    \begin{definition}[Field]
        A field is a commutative ring in which every non-zero element has a multiplicative inverse.
    \end{definition}
}

\only<3>{
    \begin{block}{Examples}
        $(\Q, +, \cdot), (\R, +, \cdot), (\C, +, \cdot), (\Z_p, +\pmod{p}, \cdot \pmod{p})$
    \end{block}
}
\end{frame}

\begin{frame}
\frametitle{Modules and Vector spaces}
\only<1>{
    \begin{definition}[Module]
        A module is an Abelian group closed under a left-right multiplication by a ring that is associative and distributive.
    \end{definition}
}
\only<1>{
    \begin{block}{Examples}
    
    \end{block}
}
\only<2>{
    \begin{definition}[Vector space]
        A vector space is a module over a field.
    \end{definition}
}
\only<2>{
    \begin{block}{Examples}
    
    \end{block}
}
\end{frame}